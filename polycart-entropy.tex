\documentclass[11pt, one side, article]{memoir}


\settrims{0pt}{0pt} % page and stock same size
\settypeblocksize{*}{33.5pc}{*} % {height}{width}{ratio}
\setlrmargins{*}{*}{1} % {spine}{edge}{ratio}
\setulmarginsandblock{.98in}{.98in}{*} % height of typeblock computed
\setheadfoot{\onelineskip}{2\onelineskip} % {headheight}{footskip}
\setheaderspaces{*}{1.5\onelineskip}{*} % {headdrop}{headsep}{ratio}
\checkandfixthelayout


\usepackage{amsthm}
\usepackage{mathtools}

\usepackage[inline]{enumitem}
\usepackage{ifthen}
\usepackage[utf8]{inputenc} %allows non-ascii in bib file
\usepackage{xcolor}

\usepackage[backend=biber, backref=true, maxbibnames = 10, style = alphabetic]{biblatex}
\usepackage[bookmarks=true, colorlinks=true, linkcolor=blue!50!black,
citecolor=orange!50!black, urlcolor=orange!50!black, pdfencoding=unicode]{hyperref}
\usepackage[capitalize]{cleveref}

\usepackage{tikz}

\usepackage{amssymb}
\usepackage{newpxtext}
\usepackage[varg,bigdelims]{newpxmath}
\usepackage{mathrsfs}
\usepackage{dutchcal}
\usepackage{fontawesome}
\usepackage{ebproof}
\usepackage{stmaryrd}


% cleveref %
  \newcommand{\creflastconjunction}{, and\nobreakspace} % serial comma
  \crefformat{enumi}{\card#2#1#3}
  \crefalias{chapter}{section}


% biblatex %
  \addbibresource{Library20220127.bib} 

% hyperref %
  \hypersetup{final}

% enumitem %
  \setlist{nosep}
  \setlistdepth{6}



% tikz %



  \usetikzlibrary{ 
  	cd,
  	math,
  	decorations.markings,
		decorations.pathreplacing,
  	positioning,
  	arrows.meta,
  	shapes,
		shadows,
		shadings,
  	calc,
  	fit,
  	quotes,
  	intersections,
    circuits,
    circuits.ee.IEC
  }
  
  \tikzset{
biml/.tip={Glyph[glyph math command=triangleleft, glyph length=.95ex]},
bimr/.tip={Glyph[glyph math command=triangleright, glyph length=.95ex]},
}

\tikzset{
	tick/.style={postaction={
  	decorate,
    decoration={markings, mark=at position 0.5 with
    	{\draw[-] (0,.4ex) -- (0,-.4ex);}}}
  }
} 
\tikzset{
	slash/.style={postaction={
  	decorate,
    decoration={markings, mark=at position 0.5 with
    	{\draw[-] (.3ex,.3ex) -- (-.3ex,-.3ex);}}}
  }
} 

\newcommand{\upp}{\begin{tikzcd}[row sep=6pt]~\\~\ar[u, bend left=50pt, looseness=1.3, start anchor=east, end anchor=east]\end{tikzcd}}

\newcommand{\bito}[1][]{
	\begin{tikzcd}[ampersand replacement=\&, cramped]\ar[r, biml-bimr, "#1"]\&~\end{tikzcd}  
}
\newcommand{\bifrom}[1][]{
	\begin{tikzcd}[ampersand replacement=\&, cramped]\ar[r, bimr-biml, "{#1}"]\&~\end{tikzcd}  
}
\newcommand{\bifromlong}[2][]{
	\begin{tikzcd}[ampersand replacement=\&, column sep=#2, cramped]\ar[r, bimr-biml, "#1"]\&~\end{tikzcd}  
}

% Adjunctions
\newcommand{\adj}[5][30pt]{%[size] Cat L, Left, Right, Cat R.
\begin{tikzcd}[ampersand replacement=\&, column sep=#1]
  #2\ar[r, shift left=7pt, "#3"]
  \ar[r, phantom, "\scriptstyle\Rightarrow"]\&
  #5\ar[l, shift left=7pt, "#4"]
\end{tikzcd}
}

\newcommand{\adjr}[5][30pt]{%[size] Cat R, Right, Left, Cat L.
\begin{tikzcd}[ampersand replacement=\&, column sep=#1]
  #2\ar[r, shift left=7pt, "#3"]\&
  #5\ar[l, shift left=7pt, "#4"]
  \ar[l, phantom, "\scriptstyle\Leftarrow"]
\end{tikzcd}
}

\newcommand{\xtickar}[1]{\begin{tikzcd}[baseline=-0.5ex,cramped,sep=small,ampersand 
replacement=\&]{}\ar[r,tick, "{#1}"]\&{}\end{tikzcd}}
\newcommand{\xslashar}[1]{\begin{tikzcd}[baseline=-0.5ex,cramped,sep=small,ampersand 
replacement=\&]{}\ar[r,tick, "{#1}"]\&{}\end{tikzcd}}



  
  % amsthm %
\theoremstyle{definition}
\newtheorem{definitionx}{Definition}[chapter]
\newtheorem{examplex}[definitionx]{Example}
\newtheorem{remarkx}[definitionx]{Remark}
\newtheorem{notation}[definitionx]{Notation}


\theoremstyle{plain}

\newtheorem{theorem}[definitionx]{Theorem}
\newtheorem{proposition}[definitionx]{Proposition}
\newtheorem{corollary}[definitionx]{Corollary}
\newtheorem{lemma}[definitionx]{Lemma}
\newtheorem{warning}[definitionx]{Warning}
\newtheorem*{theorem*}{Theorem}
\newtheorem*{proposition*}{Proposition}
\newtheorem*{corollary*}{Corollary}
\newtheorem*{lemma*}{Lemma}
\newtheorem*{warning*}{Warning}
%\theoremstyle{definition}
%\newtheorem{definition}[theorem]{Definition}
%\newtheorem{construction}[theorem]{Construction}

\newenvironment{example}
  {\pushQED{\qed}\renewcommand{\qedsymbol}{$\lozenge$}\examplex}
  {\popQED\endexamplex}
  
 \newenvironment{remark}
  {\pushQED{\qed}\renewcommand{\qedsymbol}{$\lozenge$}\remarkx}
  {\popQED\endremarkx}
  
  \newenvironment{definition}
  {\pushQED{\qed}\renewcommand{\qedsymbol}{$\lozenge$}\definitionx}
  {\popQED\enddefinitionx} 

    
%-------- Single symbols --------%
	
\DeclareSymbolFont{stmry}{U}{stmry}{m}{n}
\DeclareMathSymbol\fatsemi\mathop{stmry}{"23}

\DeclareFontFamily{U}{mathx}{\hyphenchar\font45}
\DeclareFontShape{U}{mathx}{m}{n}{
      <5> <6> <7> <8> <9> <10>
      <10.95> <12> <14.4> <17.28> <20.74> <24.88>
      mathx10
      }{}
\DeclareSymbolFont{mathx}{U}{mathx}{m}{n}
\DeclareFontSubstitution{U}{mathx}{m}{n}
\DeclareMathAccent{\widecheck}{0}{mathx}{"71}


%-------- Renewed commands --------%

\renewcommand{\ss}{\subseteq}

%-------- Other Macros --------%


\DeclarePairedDelimiter{\present}{\langle}{\rangle}
\DeclarePairedDelimiter{\copair}{[}{]}
\DeclarePairedDelimiter{\floor}{\lfloor}{\rfloor}
\DeclarePairedDelimiter{\ceil}{\lceil}{\rceil}
\DeclarePairedDelimiter{\corners}{\ulcorner}{\urcorner}
\DeclarePairedDelimiter{\ihom}{[}{]}

\DeclareMathOperator{\Hom}{Hom}
\DeclareMathOperator{\Mor}{Mor}
\DeclareMathOperator{\dom}{dom}
\DeclareMathOperator{\cod}{cod}
\DeclareMathOperator{\idy}{idy}
\DeclareMathOperator{\comp}{com}
\DeclareMathOperator*{\colim}{colim}
\DeclareMathOperator{\im}{im}
\DeclareMathOperator{\ob}{Ob}
\DeclareMathOperator{\Tr}{Tr}
\DeclareMathOperator{\el}{El}




\newcommand{\const}[1]{\texttt{#1}}%a constant, or named element of a set
\newcommand{\Set}[1]{\mathsf{#1}}%a named set
\newcommand{\ord}[1]{\mathsf{#1}}%an ordinal
\newcommand{\cat}[1]{\mathcal{#1}}%a generic category
\newcommand{\Cat}[1]{\mathbf{#1}}%a named category
\newcommand{\fun}[1]{\mathrm{#1}}%a function
\newcommand{\Fun}[1]{\mathit{#1}}%a named functor




\newcommand{\id}{\mathrm{id}}
\newcommand{\then}{\mathbin{\fatsemi}}

\newcommand{\cocolon}{:\!}


\newcommand{\iso}{\cong}
\newcommand{\too}{\longrightarrow}
\newcommand{\tto}{\rightrightarrows}
\newcommand{\To}[2][]{\xrightarrow[#1]{#2}}
\renewcommand{\Mapsto}[1]{\xmapsto{#1}}
\newcommand{\Tto}[3][13pt]{\begin{tikzcd}[sep=#1, cramped, ampersand replacement=\&, text height=1ex, text depth=.3ex]\ar[r, shift left=2pt, "#2"]\ar[r, shift right=2pt, "#3"']\&{}\end{tikzcd}}
\newcommand{\Too}[1]{\xrightarrow{\;\;#1\;\;}}
\newcommand{\from}{\leftarrow}
\newcommand{\ffrom}{\leftleftarrows}
\newcommand{\From}[1]{\xleftarrow{#1}}
\newcommand{\Fromm}[1]{\xleftarrow{\;\;#1\;\;}}
\newcommand{\surj}{\twoheadrightarrow}
\newcommand{\inj}{\rightarrowtail}
\newcommand{\wavyto}{\rightsquigarrow}
\newcommand{\lollipop}{\multimap}
\newcommand{\imp}{\Rightarrow}
\renewcommand{\iff}{\Leftrightarrow}
\newcommand{\down}{\mathbin{\downarrow}}
\newcommand{\fromto}{\leftrightarrows}
\newcommand{\tickar}{\xtickar{}}
\newcommand{\slashar}{\xslashar{}}
\newcommand{\card}{\,^{\#}}


\newcommand{\inv}{^{-1}}
\newcommand{\op}{^\tn{op}}

\newcommand{\tn}[1]{\textnormal{#1}}
\newcommand{\ol}[1]{\overline{#1}}
\newcommand{\ul}[1]{\underline{#1}}
\newcommand{\wt}[1]{\widetilde{#1}}
\newcommand{\wh}[1]{\widehat{#1}}
\newcommand{\wc}[1]{\widecheck{#1}}
\newcommand{\ubar}[1]{\underaccent{\bar}{#1}}



\newcommand{\bb}{\mathbb{B}}
\newcommand{\cc}{\mathbb{C}}
\newcommand{\nn}{\mathbb{N}}
\newcommand{\pp}{\mathbb{P}}
\newcommand{\qq}{\mathbb{Q}}
\newcommand{\zz}{\mathbb{Z}}
\newcommand{\rr}{\mathbb{R}}


\newcommand{\finset}{\Cat{Fin}}
\newcommand{\smset}{\Cat{Set}}
\newcommand{\smcat}{\Cat{Cat}}
\newcommand{\catsharp}{\Cat{Cat}^{\sharp}}
\newcommand{\ppolyfun}{\mathbb{P}\Cat{olyFun}}
\newcommand{\ccatsharp}{\mathbb{C}\Cat{at}^{\sharp}}
\newcommand{\ccatsharpdisc}{\mathbb{C}\Cat{at}^{\sharp}_{\tn{disc}}}
\newcommand{\ccatsharplin}{\mathbb{C}\Cat{at}^{\sharp}_{\tn{lin}}}
\newcommand{\ccatsharpdisccon}{\mathbb{C}\Cat{at}^{\sharp}_{\tn{disc,con}}}
\newcommand{\sspan}{\mathbb{S}\Cat{pan}}
\newcommand{\en}{\Cat{End}}

\newcommand{\List}{\Fun{List}}
\newcommand{\set}{\tn{-}\Cat{Set}}




\newcommand{\yon}{\mathcal{y}}
\newcommand{\poly}{\Cat{Poly}}
\newcommand{\dir}{\Set{Dir}}
\newcommand{\rect}{\Set{Rect}}
\newcommand{\polycart}{\poly^{\Cat{Cart}}}
\newcommand{\hh}{\mathcal{h}}
\newcommand{\ppoly}{\mathbb{P}\Cat{oly}}
\newcommand{\0}{\textsf{0}}
\newcommand{\1}{\tn{\textsf{1}}}
\newcommand{\U}{\tn{\textsf{U}}}
\newcommand{\tri}{\mathbin{\triangleleft}}
\newcommand{\R}{R}
\newcommand{\D}{T}

% lenses
\newcommand{\biglens}[2]{
     \begin{bmatrix}{\vphantom{f_f^f}#2} \\ {\vphantom{f_f^f}#1} \end{bmatrix}
}
\newcommand{\littlelens}[2]{
     \begin{bsmallmatrix}{\vphantom{f}#2} \\ {\vphantom{f}#1} \end{bsmallmatrix}
}
\newcommand{\lens}[2]{
  \relax\if@display
     \biglens{#1}{#2}
  \else
     \littlelens{#1}{#2}
  \fi
}



\newcommand{\qand}{\quad\text{and}\quad}
\newcommand{\qqand}{\qquad\text{and}\qquad}


\newcommand{\coto}{\nrightarrow}
\newcommand{\cofun}{{\raisebox{2pt}{\resizebox{2.5pt}{2.5pt}{$\setminus$}}}}

\newcommand{\coalg}{\tn{-}\Cat{Coalg}}

\newcommand{\bic}[2]{{}_{#1}\Cat{Comod}_{#2}}

% ---- Changeable document parameters ---- %

\linespread{1.1}
\allowdisplaybreaks
\setsecnumdepth{section}
\settocdepth{section}
\setlength{\parindent}{15pt}
\setcounter{tocdepth}{1}



%--------------- Document ---------------%
\begin{document}

\title{Polynomial functors and entropy}

\author{David I. Spivak}

\date{\vspace{-.2in}}

\maketitle

\begin{abstract}
Past work shows that one can associate a notion of Shannon entropy to a Dirichlet polynomial $d\coloneqq a_nn^\yon+\cdots+a_22^\yon+a_11^\yon+a_00^\yon$, regarded as an empirical distribution. Indeed, entropy can be extracted from any $d\in\dir$ by a two-step process, where the first step is a rig homomorphism out of $\dir$, the \emph{set} of Dirichlet polynomials, with rig structure given by standard addition and multiplication. In this short note, we show that this rig homomorphism can be upgraded to a rig \emph{functor}, when we replace the set of Dirichlet polynomials by the \emph{category} of ordinary (Cartesian) polynomials.

In the Cartesian case, the process has three steps. The first step is a rig functor $\polycart\to\poly$ sending a polynomial $p$ to $\dot{p}\yon$, where $\dot{p}$ is the derivative of $p$. The second is a rig functor $\poly\to\smset\times\smset\op$, sending a polynomial $q$ to the pair $(q(1),\Gamma(q))$, where $\Gamma(q)=\poly(q,\yon)$ can be interpreted as the global sections of $q$ viewed as a bundle, and $q(1)$ as its base. The last step, as for Dirichlet polynomials, is simply to extract the entropy as a real number from a pair of sets $(A,B)$; it is given by $\log A-\frac{\log B}{A}$. In summary, the entropy of $p$ is given by $H(p)=\log \dot{p}(1)-\frac{\log\Gamma(\dot{p}\yon)}{\dot{p}(1)}$.
\end{abstract}

\chapter{Introduction}

In \cite{spivak2021dirichlet} the authors---myself and Tim Hosgood---present a process for extracting the Shannon entropy from an empirical distribution, e.g.\ the following: 
\begin{equation}\label{eqn.empirical}
  \begin{tikzpicture}[scale=0.5, baseline=(pi)]
    \node at (-2,2.5) {draws};%$\dot{p}(1)\cong8\cong$};
    \begin{scope}
      \draw[rounded corners] (0,0) rectangle ++(6,5);
      \foreach \y in {1,2,3,4}
        \draw[thick,black,fill=gray] (1,\y) circle (2mm);
      \foreach \x in {2,3,4,5}
        \draw[thick,black,fill=gray] (\x,1) circle (2mm);
    \end{scope}
    \draw[thick,-Latex] (3,-0.5) to (3,-1.5);
     \node at (3.75,-1) (pi) {$\pi$};
   \node at (-2,-3) {outcomes};%$p(1)\cong5\cong$};
    \begin{scope}[shift={(0,-4)}]
      \draw[rounded corners] (0,0) rectangle ++(6,2);
      \foreach \x in {1,2,3,4,5}
        \draw[thick,black,fill=white] (\x,1) circle (2mm);
    \end{scope}
  \end{tikzpicture}
\end{equation}
The picture in \eqref{eqn.empirical} represents a statistical sample on a set of five (5) possible \emph{outcomes} and eight (8) empirical observations or \emph{draws}: four of the draws were from the first outcome, and 
the rest were evenly distributed among the remaining four outcomes. This picture corresponds to a Dirichlet polynomial $d(\yon)\coloneqq 1\cdot 4^\yon+4\cdot 1^\yon$ in the sense that $d(1)\cong 8$ and $d(0)\cong 4$. Moreover, $d$ can be regarded as a functor $d\colon\smset\op\to\smset$ and the function $\pi$ is $d$ applied to the unique function $0\to 1$. We showed that the entropy of any finite empirical distribution can be extracted after sending the corresponding Dirichlet polynomial $d$ through a certain rig homomorphism $\dir\to\rect$, where $\dir$ is the rig (in $\smset$) of Dirichlet polynomials under $+$ and $\times$, and $\rect$ is a rig that was constructed (ad hoc) for this purpose.%
\footnote{For example, addition in $\rect$ is given by sums and weighted geometric means:
\[(A_1,W_1)+(A_2,W_2)=(A_1+A_2,(W_1^{A_1}W_2^{A_2})^{\frac{1}{A_1+A_2}}).\]
An analogue of this formula will arise from the $+$-structure on $\smset\times\smset\op$; see \cref{rem.geomean}.}
In the case of \eqref{eqn.empirical}, one can calculate the entropy as 
\begin{equation}\label{eqn.example_entropy}
H(p)=H\left(\frac{1}{2},\frac{1}{8},\frac{1}{8},\frac{1}{8},\frac{1}{8}\right)=2.
\end{equation}


There is a duality \cite[Section 4]{spivak2020dirichlet} between Dirichlet polynomials and ordinary (Cartesian) polynomials.%
\footnote{Ren\`{e} Decartes at least invented the notation, e.g.\ $\yon^2+3\yon+2$, for polynomials; hence we refer to them as \emph{Cartesian polynomials} when we need to distinguish them from Dirichlet polynomials.}
 For example the Dirichlet polynomial $1\cdot 4^\yon+4\cdot 1^\yon$ corresponding to the sample in \eqref{eqn.empirical} is sent to the polynomial $1\cdot\yon^4+4\cdot\yon^1$, which we write simply as $p\coloneqq\yon^4+4\yon$. As for Dirichlet, the Cartesian polynomial $p$ encodes the base and total set of the bundle: $5\cong p(1)$ and $8\cong\dot{p}(1)$, where $\dot{p}=4\yon^3+4$ is the derivative of $p$ with respect to $\yon$. But now the map $\pi$ is inherent in $p$ and cannot simply be extracted as it was above for Dirichlet polynomials. Moreover, the map $p\mapsto\dot{p}$ is \emph{not} functorial in $p\in\poly$, nor is there generally any comparison map between $p$ and $\dot{p}$. Thus while the story Hosgood and I told about Shannon entropy for Dirichlet polynomials was also technically viable for Cartesian polynomials, we decided that the story was cleaner and seemed to have better formal properties in the Dirichlet setting.

However, there is a fix that we were not aware of, and this fix makes the Cartesian story significantly \emph{more categorical} than the Dirichlet story. Indeed, the \emph{function} $\dir\to\rect$ is replaced by a \emph{functor} $\hh\colon\polycart\to\smset\times\smset\op$, described in \cref{sec.CT}. Both the function and the functor are rig maps, i.e.\ they both preserve a notion of addition and multiplication, where the notion of multiplication is given by a what appears to be a novel monoidal product on  $\smset\times\smset\op$, namely
\[
	(A_1,B_1)\otimes(A_2,B_2)\coloneqq\left(A_1A_2\,,\,B_1^{A_2}B_2^{A_1}\right).
\]
This monoidal product $\otimes$ distributes over the coproduct, making $\smset\times\smset\op$ a distributive monoidal category, and in particular a \emph{rig category}.

There are two pieces left to describe: the functor $\polycart\To{\hh}\smset\times\smset\op$ and the partial function $\ob(\smset\times\smset\op)\To{L}\rr$ that extracts the entropy. The latter is roughly the following formula, which is defined precisely in \cref{sec.entropy}:
\[
L(A,B)\coloneqq\log A - \log \sqrt[A]{B}.
\]
In \cref{rem.log_aspect_ratio} we will give some intuition for this function in the context of polynomials, but for now we can think of $\log A$ as the amount of information possible in $A$ and $\log\sqrt[A]{B}=\frac{\log B}{A}$ as the amount of information lost to inefficiencies in the encoding.

The majority of our attention in this paper is focused on the rig functor $\hh\colon\polycart\to\smset\times\smset\op$, which can be factored into two parts:
\[
\polycart\To{p\mapsto\dot{p}\yon}\poly\To{q\mapsto\big(q(1),\Gamma(q)\big)}\smset\times\smset\op,
\]
where $\Gamma(q)\coloneqq\poly(q,\yon)$. We will explain these two functors and all the notation involved when we get to \cref{chap.background,chap.main}. We are primarily interested in the fact that each of the above functors preserves the rig structure (i.e.\ each is strong monoidal with respect to $+$ and $\otimes$) and together define a formal root $\hh(p)\in\smset\times\smset\op$ for any $p\in\polycart$.

We conclude our summary by giving the full composite: the function that takes a polynomial $p$ and returns the entropy of the corresponding empirical distribution is given by
\[
  H(p)\coloneqq\log \dot{p}(1)-\frac{\log\Gamma(\dot{p}\yon)}{\dot{p}(1)}
\]
For example, consider the polynomial $p=\yon^4+4\yon$, depicted in \eqref{eqn.empirical}. Then we calculate
\[
\dot{p}\yon=4\yon^4+4\yon
,\quad
\dot{p}(1)=8
,\quad
\Gamma(\dot{p}\yon)=4^4*1^4=2^8
,\qand 
H(p)=\log 8-\frac{\log 2^8}{8}=2
\]
which agrees with the calculation in \eqref{eqn.example_entropy}. In the above terms, there are $\log\dot{p}(1)=3$ bits of information possible in a sample size of $8$, but $\log\sqrt[8]{2^8}=1$ bit lost to inefficiencies in the encoding given by \eqref{eqn.empirical}.

The paper is fairly short, and the introduction was written so that people familiar with $\poly$ may not need to read further. For less familiar readers, we include two additional sections: a section giving the relevant background on $\poly$ and $\polycart$, and a section proving the main results.

There have been other categorical approaches to entropy, most notably \cite{baez2011characterization}, \cite{baez2014bayesian}, and \cite{leinster2021entropy}. Our presentation here has almost nothing in common with those; as mentioned above, it is closely aligned with \cite{spivak2021dirichlet}.

\section*{Acknowledgments}
This material is based upon work supported by the Air Force Office of Scientific Research under award number FA9550-20-1-0348.

\chapter{Background on polynomial functors}\label{chap.background}

Readers familiar with the rig category $(\poly,0,+,\yon,\otimes)$ should skip to \cref{sec.CT}.

%---- Section ----%
\section{Basics}


The main purpose of this section is to fix notation and provide a brief overview of polynomial functors in one variable. More extensive background material can be found in \cite{spivak2022poly} and \cite{kock2012polynomial}. 

\begin{definition}[Polynomial functor]\label{def.poly}
Given a set $S$, we denote the corresponding representable functor by
\[\yon^S\coloneqq\smset(S,-)\colon\smset\to\smset,\]
e.g. $\yon^S(X)\coloneqq X^S$. In particular $\yon=\yon^1$ is the identity and $\yon^0=1$ is constant singleton.

A \emph{polynomial functor} is a functor $p\colon\smset\to\smset$ that is isomorphic to a sum of representables, i.e.\ for which there exists a set $T$, a set $p[t]\in\smset$ for each $t\in T$, and an isomorphism of functors
\[
p\cong\sum_{t\in T}\yon^{p[t]}.
\]
We refer to $T$ as the set of \emph{$p$-types}, and for each type $t\in T$ we refer to $p[t]$ as the set of \emph{$p$-terms of type $t$}.%

A \emph{morphism} $\varphi\colon p\to p'$ of polynomial functors is simply a natural transformation between them. It is called \emph{cartesian} if for every map of sets $f\colon S\to S'$, the naturality square
\[
\begin{tikzcd}
  p(S)\ar[r, "p(f)"]\ar[d, "\varphi(S)"']&p(S)\ar[d, "\varphi(S')"]\\
  p'(S')\ar[r, "p'(f)"']&p'(S')\ar[ul, phantom, very near end, "\lrcorner"]
\end{tikzcd}
\]
is a pullback of sets. We denote the category of polynomial functors by $\poly$ and the wide subcategory of polynomials and cartesian maps by $\polycart\ss\poly$.\end{definition}

For any polynomial $p=\sum_{t\in T}\yon^{p[t]}$, we have a canonical isomorphism $p(1)\cong T$; hence from now on we will denote $p$ by
\begin{equation}\label{eqn.poly_notation}
p=\sum_{I\in p(1)}\yon^{p[I]}
\end{equation}
so that each $p$-types is written with an upper-case letter, e.g. $I\in p(1)$, and its terms are written with corresponding lower-case letters, e.g. $i\in p[I]$.

\begin{remark}\label{rem.positions_and_directions}
Using the Yoneda lemma, the fact that a morphism in $\poly$ is just a natural transformation, and the fact that a polynomial is a coproduct of representables, we derive
\begin{align*}
	\poly(p,q)&=
	\poly\left(\sum_{I\in p(1)}\yon^{p[I]},\sum_{J\in p(1)}\yon^{q[J]}\right)\\&\cong
	\prod_{I\in p(1)}\poly\left(\yon^{p[I]},\sum_{J\in p(1)}\yon^{q[J]}\right)\\&\cong
	\prod_{I\in p(1)}\sum_{J\in q(1)}\smset(q[J],p[I]).
\end{align*}
Thus we can understand a morphism $p\to q$ in $\poly$ to consist of two parts $(\varphi_1,\varphi^\sharp)$ as follows:
\begin{equation}\label{eqn.mapsharp}
  \varphi_1\colon p(1)\to q(1)
  \qqand
  \varphi^\sharp_I\colon q[J]\to p[I],
\end{equation}
where $J\coloneqq\varphi_1(I)$. That is, $\varphi_1$ is a function from $p$-types to $q$-types, and $\varphi^\sharp_i$ is a function on terms that \emph{depends on a choice of position $I\in p(1)$}. We refer to $\varphi_1$ as the \emph{on-types function} and to $\varphi^\sharp$ as the \emph{backwards on-terms} function. 

One can check that a map $\varphi\colon p\to q$ is cartesian iff the backwards-on-terms function $\varphi^\sharp_I$ is a bijection $p[I]\cong q[\varphi_1I]$ for each type $I\in p(1)$.
\end{remark}

\begin{example}[Types and global sections, $p(1)$ and $\Gamma(p)$]\label{ex.pos_glob}
For any polynomial $p$, we will be particularly interested in two sorts of maps: $\yon\to p$ and $p\to\yon$. The former is easy: a map $\yon\to p$ is given on types by choosing a single type $I\in p(1)$ to be the image of the unique type $!\in\yon(1)$ and given backward on terms using the unique choice of function $p[I]\to 1=\yon[!]$. Thus we have $p(1)\cong\poly(\yon,p)$.

More interesting are the maps $\gamma\colon p\to\yon$. This time $\gamma$ is trivial on types: each type $I\in p(1)$ is sent to the unique type $!\in\yon(1)$. However on terms, we need a map $\varphi^\sharp_I\colon 1\to p[I]$ for each $I$, meaning a choice of term $i\in I$ for each $I\in p(1)$. In other words, writing $\Gamma(p)\coloneqq\poly(p,\yon)$, we have
\begin{equation}\label{eqn.global}
\Gamma(p)\cong\prod_{I\in p(1)}p[I].
\end{equation}
We refer to $\Gamma(p)$ as the set of \emph{global sections} of $p$, as is justified by the bundle terminology the next section.

Note that $-(1)\colon\poly\to\smset$ and $\Gamma\colon\poly\to\smset\op$ are functorial, as they are represented and corepresented by $\yon\in\poly$. We will be very interested in the functor
\begin{equation}\label{eqn.fundamental}
\R\colon\poly\to\smset\times\smset\op
\end{equation}
given by $\R(p)\coloneqq (p(1),\Gamma(p))$. In fact, $\R$ is a left adjoint, but we do not need that for this paper. In \cref{rem.geomean} we will explain that $\R(p)$ can be viewed as the \emph{rectangular aspect} of the polynomial $p$, hence the name $\R$.
\end{example}

%\begin{definition}[Polynomial functor]\label{def.poly}
%Given a set $S$, we denote the corresponding representable functor by
%\[\yon^S\coloneqq\smset(S,-)\colon\smset\to\smset,\]
%e.g. $\yon^S(X)\coloneqq X^S$. In particular $\yon=\yon^1$ is the identity and $\yon^0=1$ is constant singleton.
%
%A \emph{polynomial functor} is a functor $p\colon\smset\to\smset$ that is isomorphic to a sum of representables, i.e.\ for which there exists a set $I$, a set $p[i]\in\smset$ for each $i\in I$, and an isomorphism
%\[
%p\cong\sum_{i\in I}\yon^{p[i]}.
%\]
%We call $I$ the set of \emph{$p$-positions}, and for each type $i\in I$ we call $p[i]$ the set of \emph{$p$-directions at $i$}.%
%
%A \emph{morphism} $\varphi\colon p\to p'$ of polynomial functors is simply a natural transformation between them. A morphism is called \emph{cartesian} if for every map of sets $f\colon S\to S'$, the naturality square
%\[
%\begin{tikzcd}
%  p(S)\ar[r, "p(f)"]\ar[d, "\varphi(S)"']&p(S)\ar[d, "\varphi(S')"]\\
%  p'(S')\ar[r, "p'(f)"']&p'(S')\ar[ul, phantom, very near end, "\lrcorner"]
%\end{tikzcd}
%\]
%is a pullback of sets. We denote the category of polynomial functors by $\poly$ and the wide subcategory of polynomials and cartesian maps by $\polycart$.
%\end{definition}
%
%For any polynomial $p=\sum_{i\in I}\yon^{p[i]}$, we have a canonical isomorphism $p(1)\cong I$; hence from now on we will denote $p$ by
%\begin{equation}\label{eqn.poly_notation}
%p=\sum_{i\in p(1)}\yon^{p[i]}.
%\end{equation}
%
%\begin{remark}\label{rem.positions_and_directions}
%Using the Yoneda lemma, we can understand a morphism $p\to q$ in $\poly$ to consist of two parts $(\varphi_1,\varphi^\sharp)$ as follows:
%\begin{equation}\label{eqn.mapsharp}
%  \varphi_1\colon p(1)\to q(1)
%  \qqand
%  \varphi^\sharp_i\colon q[j]\to p[i],
%\end{equation}
%where $j\coloneqq\varphi_1(i)$. That is, $\varphi_1$ is a function from $p$-positions to $q$-positions, and $\varphi^\sharp_i$ is a function on directions that \emph{depends on a choice of position $I\in p(1)$}. We refer to $\varphi_1$ as the \emph{on-positions function} and to $\varphi^\sharp$ as the \emph{backward on-directions} function.
%
%One can check that a map $\varphi\colon p\to q$ is cartesian iff the backward-on-directions function $\varphi^\sharp_i$ is a bijection $p[I]\cong q[\varphi_1I]$ for each type $I\in p(1)$.
%\end{remark}

\section{Derivatives and bundles}

We can understand polynomial functors in terms of bundles, using the derivative. For any polynomial $p$, its derivative $\dot{p}$ is defined as follows:
\begin{equation}\label{eqn.dotp}
\dot{p}\coloneqq\sum_{I\in p(1)}\sum_{i\in p[I]}\yon^{p[I]-\{i\}}
\end{equation}
where $p[I]-\{i\}$ denotes the set-difference. Note that $\dot{p}(1)\cong\sum_{I\in p(1)}p[I]$ is the set of all $p$-terms, and it comes with a map $\dot{p}(1)\to p(1)$ to the set of $p$-types. Often in the literature, this map of sets---which we call a bundle---is taken to be the polynomial itself. A map of polynomials $\varphi\colon p\to q$ can be written in terms of these bundles:
\[
\begin{tikzcd}
	\dot{p}(1)\ar[d]&p(1)\times_{q(1)}\dot{q}(1)\ar[l, "\varphi^\sharp"']\ar[r]\ar[d]&\dot{q}(1)\ar[d]\\
	p(1)\ar[r, equal]&p(1)\ar[r, "\varphi_1"']&q(1)\ar[ul, phantom, very near end, "\lrcorner"]
\end{tikzcd}
\]
Just as in \cref{rem.positions_and_directions}, one provides a forward map on types $\varphi_1\colon p(1)\to q(1)$, at which point one takes the pullback of that map with $\dot{q}(1)\to q(1)$, and then one provides a backward map $\varphi^\sharp\colon p(1)\times_{q(1)}\dot{q}(1)\to \dot{p}(1)$ on directions.
Again, $\varphi$ is cartesian iff $\varphi^\sharp$ is a bijection.

\begin{proposition}\label{prop.dotpy}
The assignment $p\mapsto\dot{p}\yon$ is a functor $\polycart\to\polycart$.
\end{proposition}
\begin{proof}
We can think of $\dot{p}\yon$ as follows:
\begin{equation}\label{eqn.derivy}
\dot{p}\yon\cong\sum_{I\in p(1)}\sum_{i\in p[I]}\yon^{p[I]}
\end{equation}
Given a cartesian map $\varphi\colon p\to q$, the bijection $\varphi^\sharp\colon q[\varphi_1(I)]\cong p[I]$ lets us define a map $\dot{p}\yon\to\dot{q}\yon$ in an obvious way. 
\end{proof}

\begin{remark}\label{rem.comonad}
In fact, the assignment $(p\mapsto\dot{p}\yon)\colon\polycart\to\polycart$ extends to a comonad on $\polycart$. The counit map $\epsilon_p\colon\dot{p}\yon\to p$ is cartesian and is given on types by $(I,i)\mapsto I$. The comultiplication $\delta_p\colon\dot{p}\yon\to\ddot{p}\yon^2+\dot{p}\yon$ is given by the coproduct inclusion.

A coalgebra for this comonad is a polynomial $p$ equipped with a map $\gamma\colon p\to\dot{p}\yon$ such that $\epsilon_p\circ\gamma=\id_p$; it is not hard to check that the other condition holds for free. Hence a coalgebra structure on $p$ can be identified with a choice a global section $p\to\yon$, i.e.\ an element $\gamma\in\Gamma(p)$. Of course the map $p\to\yon$ is not cartesian in general, so the only way it can be encoded in $\polycart$ is via this coalgebra structure. A map of coalgebras is a cartesian map $\varphi\colon p\to p'$ that commutes with the global sections: $\Gamma(\varphi)(\gamma')=\gamma$.

The above is intriguing in that $\Gamma(p)$ is a major player in the story of this paper, but we currently know of no further connection between entropy and this comonad. 
\end{remark}

\section{Rig monoidal structure on $\poly$}

The category $\poly$ has coproducts $p+q$ and products $p\times q$ given by usual polynomial arithmetic. We will be more interested in the former:%
\footnote{
The only reason we introduce $\times$ for $\poly$ is that we will be writing $\dot{p}\yon$, which is shorthand for $\dot{p}\times\yon$, throughout the paper. We will not see other products of polynomials, though products of sets will come up quite often.
}
coproducts constitute a symmetric monoidal product with unit $0$. A type in $p+q$ is a type in $p$ or disjointly a type in $q$, and a term of that type is as specified in $p$ or $q$, respectively.

We will also be interested in another monoidal product called \emph{Dirichlet product} and denoted $-\otimes-$; the types and terms of $p\otimes q$ are given by the following formula:
\begin{equation}\label{eqn.dir_formula}
  \left(\sum_{I\in p(1)}\yon^{p[I]}\right)\otimes
  \left(\sum_{J\in q(1)}\yon^{q[J]}\right)\coloneqq
  \sum_{(I,J)\in p(1)\times q(1)}\yon^{p[I]\times q[J]}.
\end{equation}
This gives a symmetric monoidal structure $(\poly,\yon,\otimes)$. A type in $p\otimes q$ is just a pair of types $(I,J)\in p(1)\times q(1)$ and a term of it is just a pair of terms $(i,j)\in p[I]\times q[J]$.

In the language of bundles, $p+q$ and $p\otimes q$ are respectively given by
\[
\begin{tikzcd}
	\dot{p}(1)+\dot{q}(1)\ar[d]&\dot{p}(1)\times\dot{q}(1)\ar[d]\\
	p(1)+q(1)&p(1)\times q(1)
\end{tikzcd}
\]
i.e. $\dot{(p+q)}(1)\cong\dot{p}(1)+\dot{q}(1)$ and $\dot{(p\otimes q)}(1)\cong\dot{p}(1)\times\dot{q}(1)$.

The $\otimes$-structure distributes over the $+$ structure:
\[
p\otimes (q_1+q_2)\cong (p\otimes q_1)+(p\otimes q_2),
\]
thus making $(\poly,0,+,\yon,\otimes)$ a distributive monoidal category, and in particular a rig monoidal category.

\begin{remark}[Leibniz and chain rules]\label{rem.leibniz}
Some readers may be interested in the Leibniz rule and chain rule, that
\begin{align*}
	\dot{(p\times q)}&\cong\dot{p}\times q+p\times\dot{q}\\
	\dot{(p\tri q)}&\cong(\dot{p}\tri q)\times\dot{q}
\end{align*}
where $\times$ is the categorical product and $\tri$ is the composition product in $\poly$. These hold, but we will not need them in this paper.
\end{remark}

\chapter{Main results}\label{chap.main}

We divide this section into two parts. \cref{sec.CT} is the category theory part, in which we provide what seems to be a novel symmetric monoidal structure on $\smset\times\smset\op$ and show that both $p\mapsto\dot{p}\yon$ and $q\mapsto(q(1),\Gamma(q))$ are rig functors. At the end of this section, we will have a rig functor $\hh\colon\polycart\to\smset\times\smset\op$ that does the categorical work of Shannon entropy.

\cref{sec.entropy} is the finishing step, providing a function $\ob(\smset\times\smset\op)\to\rr$ and showing that when it is combined with the above, the map $H\colon\ob(\polycart)\to\rr$ sends an appropriately finite polynomial $p$ to the Shannon entropy of the empirical distribution defined by $p$.

\section{Categorical entropy of a polynomial}\label{sec.CT}

Below we will often denote products merely by juxtaposition, both for polynomials $p\yon\coloneqq p\times \yon$ and for sets $AB\coloneqq A\times B$. Recall the functor $p\mapsto\dot{p}\yon$ from \cref{prop.dotpy}.

\begin{proposition}
The functor $p\mapsto\dot{p}\yon$ is a rig functor $\polycart\to\polycart$. 
\end{proposition}
\begin{proof}
Clearly $\dot{0}=0$ and $\dot{(p+q)}\cong\dot{p}+\dot{q}$, and by multiplying both sides by $\yon$ we see that the functor $p\mapsto p\yon$ preserves the coproduct structure. There is an isomorphism $\dot{\yon}\yon\cong\yon$, and for any $p,q\in\polycart$ there is also an isomorphism $\dot{(p\otimes q)}\yon\cong(\dot{p}\yon)\otimes(\dot{q}\yon)$, as follows from \eqref{eqn.derivy} and \eqref{eqn.dir_formula}; thus $p\mapsto\dot{p}\yon$ preserves the $\otimes$-structure. All of these isomorphisms are natural in $p,q\in\polycart$, completing the proof.
\end{proof}

The following corollary is straightforward, since $\polycart$ inherits $+$ and $\otimes$ from the forgetful functor $\polycart\to\poly$.

\begin{corollary}\label{cor.D}
The functor $\D(p)\coloneqq\dot{p}\yon$ is a rig functor $\D\colon\polycart\to\poly$. 
\end{corollary}

\begin{remark}[Total polynomial]\label{rem.total}
Note that for any $p$ we have $(\dot{p}\yon)(1)\cong\dot{p}(1)$. We think of $\dot{p}\yon$ as the \emph{total polynomial} of $p$; e.g.\ a section of the map $\dot{p}\yon\to p$ from \cref{rem.comonad} is the same as a section $\gamma\in\Gamma(p)$.
\end{remark}


\begin{example}\label{ex.global_D}
For any polynomial $p$, we have
\[
\Gamma(\dot{p}\yon)\cong\prod_{I\in p(1)}p[I]^{p[I]}.
\]
This formula---which follows directly from \cref{eqn.global,eqn.derivy}---will be relevant when connecting the category theory to Shannon entropy later on. 
\end{example}

\begin{proposition}
The category $\smset\times\smset\op$ has a distributive monoidal structure:
\begin{align}
  (A_1,B_1)+(A_2,B_2)&\coloneqq(A_1+A_2\,,\,B_1B_2)
  \label{eqn.sum_geomean}\\\label{eqn.prod_prod}
  (A_1,B_1)\otimes(A_2,B_2)&\coloneqq(A_1A_2\,,\,B_1^{A_2}B_2^{A_1})
\end{align}
The units are $(0,1)$ and $(1,1)$ respectively.
\end{proposition}
\begin{proof}
Coproducts in $\smset\op$ are products in $\smset$, justifying the first line; these clearly form a symmetric monoidal structure. For the $\otimes$-monoidal structure, note that the formula is functorial in $A\in\smset$ and $B\in\smset\op$. It is also symmetric as well as unital: $(1,1)\otimes(A_2,B_2)\cong(A_2,B_2)$. Associativity is justified as follows:
\begin{align*}
  (A_1,B_1)\otimes((A_2,B_2)\otimes(A_3,B_3))&\cong
  %(A_1(A_2A_3),B_1^{A_2A_3}(B_2^{A_3}B_3^{A_2})^{A_1})\\\cong
	(A_1A_2A_3,B_1^{A_2A_3}B_2^{A_1A_3}B_3^{A_1A_2})\\&\cong
	((A_1,B_1)\otimes(A_2,B_2))\otimes(A_3,B_3).
\end{align*}
There is an absorption map $(0,1)\otimes (A,B)\cong(0,B)\to(0,1)$, and the distributivity of $\otimes$ over $+$ is justified as follows:
\begin{align*}
  (A,B)\otimes\big((A_1,B_1)+(A_2,B_2)\big)&\cong 
  \big(A(A_1+A_2),B^{A_1+A_2}(B_1B_2)^A\big)\\&\cong
  \big(AA_1+AA_2,B^{A_1}B^{A_2}B_1^AB_2^A\big)\\&\cong
  \big((A,B)\otimes(A_1,B_1))+((A,B)\otimes(A_2,B_2)\big).
\end{align*}
We leave the remaining details to the interested reader.
\end{proof}

\begin{remark}[Formal roots and rectangular aspect]\label{rem.geomean}
One can think of an object $(A,B)\in\smset\times\smset\op$ as formally representing the $A$th root of $B$, i.e.\ the number $\sqrt[A]{B}=B^{\frac{1}{A}}$, keeping track of the base $A$ as well. From this perspective, the sum from \eqref{eqn.sum_geomean} represents the geometric mean and the monoidal product from \eqref{eqn.prod_prod} represents the product:
\[
(B_1B_2)^\frac{1}{A_1+A_2}=\left(\left(B_1^\frac{1}{A_1}\right)^{A_1}\times \left(B_2^\frac{1}{A_2}\right)^{A_2}\right)^{\frac{1}{A_1+A_2}}
\qqand
(B_1^{A_2}B_2^{A_1})^\frac{1}{A_1A_2}=B_1^\frac{1}{A_1}B_2^\frac{1}{A_2}.
\]
We can think of $(A,B)$ as representing a rectangle with length $A$ and width $\sqrt[A]{B}$.

For any polynomial $p$, the functor $\R(p)\coloneqq(p(1),\Gamma(p))$ from \eqref{eqn.fundamental} is consonant with this interpretation. We may say that $\R(p)$ denotes the \emph{rectangular aspect} of $p$ in the sense that $p(1)$ represents the length and $\sqrt[p(1)]{\Gamma(p)}$, the geometric mean of the fiber cardinalities, represents the width.
\end{remark}

\begin{proposition}\label{prop.W}
The functor $\R\colon\poly\to\smset\times\smset\op$ from \eqref{eqn.fundamental} is a rig functor.
\end{proposition}
\begin{proof}
Recall from \eqref{eqn.fundamental} that $\R(p)\coloneqq(p(1),\Gamma(p))$. Clearly $0(1)=0$ and $(p+q)(1)\cong p(1)+q(1)$. Also $\Gamma(0)=1$ and $\Gamma(p+q)\cong\Gamma(p)\times\Gamma(q)$; hence $\R$ preserves the $(0,+)$ monoidal structure. Moreover, we have $\yon(1)=1$ and $(p\otimes q)(1)\cong p(1)\times q(1)$ and $\Gamma(\yon)= 1$, so to show that $\R$ preserves the $(\yon,\otimes)$ monoidal structure, it remains only to provide an isomorphism 
\[
  \Gamma(p\otimes q)\cong \Gamma(p)^{q(1)}\times\Gamma(q)^{p(1)}.
\]
It is given as follows:
\begin{align*}
	\Gamma(p\otimes q)&\cong
	\prod_{(I,J)\in p(1)\times q(1)}p[I]q[J]\\&\cong
	\left(\prod_{(I,J)\in p(1)\times q(1)}p[I]\right)\times
		\left(\prod_{(I,J)\in p(1)\times q(1)}q[J]\right)\\&\cong
	\prod_{J\in q(1)}\prod_{I\in p(1)}p[I]\times\prod_{I\in p(1)}\prod_{J\in q(1)}q[J]\\&\cong
	\Gamma(p)^{q(1)}\times\Gamma(q)^{p(1)}	
\end{align*}
\end{proof}

\begin{remark}[Products on $\smset\times\smset\op$]
In fact $\smset\times\smset\op$ has categorical products: the unit is $(1,0)$ and the product is
\[(A_1,B_1)\times(A_2,B_2)\coloneqq(A_1A_2,B_1+B_2).\]
All functors between categories with products have an op-lax structure, hence $\R\colon(\poly,1,\times)\to(\smset\times\smset\op,(1,0),\times)$ does too:
 \[
 \R(1)\cong(1,0)\qqand\R(p\times q)\to\R(p)\times\R(q).
 \]
 But $R$ is no better-behaved than that, and since $\dot{p}$ only satisfies a Leibniz rule with respect to product (see \cref{rem.leibniz})---i.e.\ since it does not actually \emph{preserve} products---we see no way to use the $\times$ structure on $\poly$ or $\smset\times\smset\op$ in this paper.
\end{remark}


We summarize the above section before we go on to the next. Namely, the functors $\D\colon\polycart\to\poly$ and $\R\colon\poly\to\smset\to\smset\op$ from \cref{cor.D,prop.W} compose to form a rig functor $\hh\coloneqq \R\circ \D$ given by
\begin{equation}\label{eqn.entropy_data}
\begin{aligned}
	\polycart&\To{\mathcal{h}}\smset\times\smset\op\\
	p&\mapsto (\dot{p}(1),\Gamma(\dot{p}\yon)).
\end{aligned}
\end{equation}

We refer to $\hh(p)\in\smset\times\smset\op$ as \emph{the categorical entropy} of the polynomial $p$. These two sets have left behind any semblance of the probability distribution associated with $p$, but they contain everything one needs in order to compute $p$'s entropy---as we'll see in \cref{thm.main}---and they are rig-functorial in $p$. 

\section{Shannon entropy}\label{sec.entropy}

We define a partial function $L\colon\ob(\smset\times\smset\op)\to\rr$ by
\begin{equation}\label{eqn.L}
(A,B)\mapsto \log A-\frac{\log B}{A}.
\end{equation}
Equivalently, $L(A,B)=\log A-\log\sqrt[A]{B}$. When $A=0$ and $B=1$, we define this function to be $L(0,1)=0$; for all cases where $A=0$, or $B=0$, or either $A$ or $B$ is infinite, we leave $L(A,B)$ undefined. We will be only interested in this map when it is composed with the categorical entropy $\mathcal{h}$ from \eqref{eqn.entropy_data}, and \cref{lemma.dontworry} below says that we do not need to worry about the undefined cases.

\begin{lemma}\label{lemma.dontworry}
Let $p\in\polycart$ with categorical entropy $(A,B)\coloneqq\hh(p)$, and suppose that $\card\dot{p}(1)<\infty$. Then we have that
\begin{enumerate}[label=\roman*.]
	\item $B\neq 0$,
	\item if $A=0$ then $B=1$, and
	\item both $A$ and $B$ are finite.
\end{enumerate}
\end{lemma}
\begin{proof}
By definition of $\hh$, we have that $A\coloneqq\dot{p}(1)$ and $B\coloneqq\Gamma(\dot{p}\yon)$. 
\begin{enumerate}[label=\roman*.]
	\item One easily checks using \eqref{eqn.global} that for any $q\in\poly$, the set $\Gamma(q\yon)\neq0$ is nonempty since every $(q\yon)$-type has at least one term. 
	\item If $\dot{p}(1)=0$ then $p\in\smset$ is constant, so $\dot{p}\yon=0$ as well, and $\Gamma(0)=1$ by \eqref{eqn.global}.
	\item By assumption $\card A=\card\dot{p}(1)<\infty$. For $B$, note that there are only a finite number of $I\in p(1)$ for which $p[I]$ is nonempty, so by \eqref{eqn.global} and \eqref{eqn.dotp} the set $\Gamma(\dot{p}\yon)$ is finite.
\qedhere
\end{enumerate}
\end{proof}

\begin{remark}[Log aspect ratio]\label{rem.log_aspect_ratio}
With the interpretation of an object $(A,B)\in\smset\times\smset\op$ as a rectangle with length $A$ and width $\sqrt[A]{B}$, as in \cref{rem.geomean}, we can think of $L(A,B)=\log A -\log\sqrt[A]{B}$ as its \emph{log aspect ratio}, a quantity that has come up in the study of vision \cites{talbot2011arc,dickinson2017separate}, though we're not claiming that this connection is meaningful.
\end{remark}

\begin{definition}[Empirical distribution]
Let $p\neq 0$ be a nonzero polynomial and suppose that the cardinality of $\dot{p}(1)\in\smset$ is finite, $\card\dot{p}(1)<\infty$. We define the \emph{empirical distribution defined by $p$} to be the following function $P\colon p(1)\to[0,1]$:
\[
 P(I)\coloneqq\frac{\card p[I]}{\card\dot{p}(1)}
\]
Note that $1=\sum_{I\in p(1)}P(I)$, so this is a probability distribution.
\end{definition}

\begin{example}[Outcomes, samples, and draws]
Consider an experiment consisting of a coin; it has two outcomes, heads and tails. In this situation a sample consists of a set of draws, i.e.\ coin-flips. The polynomial $p$ corresponds to the sample: the set of outcomes is $p(1)$---in this case $p(1)=\{\text{heads, tails}\}$ has two elements---and for each outcome $i\in p(1)$ the set of draws with outcome $i$ is $p[i]$. The set $\dot{p}(1)$ is the total number of observations.
\end{example}

Recall that the Shannon entropy $H(P)$ of a probability distribution $P\colon X\to[0,1]$ is given by
\[
H(P)\coloneqq-\sum_{x\in X}P(x)\log P(x).
\]

The following theorem could be summarized as follows: ``thinking of $p\in\polycart$ as a statistical sample, the entropy $H(P)$ of the corresponding probability distribution $P$ is equal to the log ratio of the rectangular aspect of $p$'s total polynomial''; see Remarks~\ref{rem.total},~\ref{rem.geomean},~and~\ref{rem.log_aspect_ratio}.

\begin{theorem}\label{thm.main}
Let $p\neq0$ be a nonzero polynomial with $\card\dot{p}(1)<\infty$, and let $P$ be the empirical distribution defined by $p$. Then the following equation holds
\[
H(P)=L(\hh(p))
\]
where $H$ is the Shannon entropy and $L,\hh$ are as defined in \cref{eqn.entropy_data,eqn.L}.
\end{theorem}
\begin{proof}
We need to show that the following holds:
\[
H(P)=\log\dot{p}(1)-\frac{\log\Gamma(\dot{p}\yon)}{\dot{p}(1)}.
\]
With the fact $\Gamma(\dot{p}\yon)\cong\prod_{I\in p(1)}p[I]^{p[I]}$ from \cref{ex.global_D} in hand, this is a routine calculation:
\begin{align*}
	H(P)&\coloneqq
	-\sum_{I\in p(1)}\frac{\card p[I]}{\card\dot{p}(1)}\log \frac{\card p[I]}{\card\dot{p}(1)}\\&=
	\frac{1}{\card\dot{p}(1)}\sum_{I\in p(1)}\card p[I]\big(\log\card\dot{p}(1)-\log\card p[I]\big)\\&=
	\frac{1}{\card\dot{p}(1)}\left(\card\dot{p}(1)\log\card\dot{p}(1)-\log\prod_{I\in p(1)}\card p[I]^{\card p[I]}\right)\\&=
	\log\card\dot{p}(1)-\frac{\log\Gamma(\dot{p}\yon)}{\dot{p}(1)}
\end{align*}
\end{proof}

\begin{example}[Entropy of a uniform distribution]
It is well-known and easy to calculate that if $P$ is a uniform distribution on $A$ elements, then $H(p)=\log(A)$. There are many samples that correspond to $P$; they differ in their number of observations. The sample in which $AB$-many observations are taken---each outcome occurring $B$-many times---corresponds to the polynomial $A\yon^B$.

Our formula for entropy needs to agree, and it does: $\hh(p)\cong(AB,B^{AB})$, and
\[
L(\hh(p))=\log(AB)-\frac{\log(B^{AB})}{AB}=\log A.
\qedhere
\]
\end{example}


\printbibliography 
\end{document}
