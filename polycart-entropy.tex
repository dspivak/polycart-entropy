\documentclass[11pt, one side, article]{memoir}


\settrims{0pt}{0pt} % page and stock same size
\settypeblocksize{*}{34.5pc}{*} % {height}{width}{ratio}
\setlrmargins{*}{*}{1} % {spine}{edge}{ratio}
\setulmarginsandblock{.98in}{.98in}{*} % height of typeblock computed
\setheadfoot{\onelineskip}{2\onelineskip} % {headheight}{footskip}
\setheaderspaces{*}{1.5\onelineskip}{*} % {headdrop}{headsep}{ratio}
\checkandfixthelayout


\usepackage{amsthm}
\usepackage{mathtools}

\usepackage[inline]{enumitem}
\usepackage{ifthen}
\usepackage[utf8]{inputenc} %allows non-ascii in bib file
\usepackage{xcolor}

\usepackage[backend=biber, backref=true, maxbibnames = 10, style = alphabetic]{biblatex}
\usepackage[bookmarks=true, colorlinks=true, linkcolor=blue!50!black,
citecolor=orange!50!black, urlcolor=orange!50!black, pdfencoding=unicode]{hyperref}
\usepackage[capitalize]{cleveref}

\usepackage{tikz}

\usepackage{amssymb}
\usepackage{newpxtext}
\usepackage[varg,bigdelims]{newpxmath}
\usepackage{mathrsfs}
\usepackage{dutchcal}
\usepackage{fontawesome}
\usepackage{ebproof}
\usepackage{stmaryrd}


% cleveref %
  \newcommand{\creflastconjunction}{, and\nobreakspace} % serial comma
  \crefformat{enumi}{\card#2#1#3}
  \crefalias{chapter}{section}


% biblatex %
  \addbibresource{Library20220127.bib} 

% hyperref %
  \hypersetup{final}

% enumitem %
  \setlist{nosep}
  \setlistdepth{6}



% tikz %



  \usetikzlibrary{ 
  	cd,
  	math,
  	decorations.markings,
		decorations.pathreplacing,
  	positioning,
  	arrows.meta,
  	shapes,
		shadows,
		shadings,
  	calc,
  	fit,
  	quotes,
  	intersections,
    circuits,
    circuits.ee.IEC
  }
  
  \tikzset{
biml/.tip={Glyph[glyph math command=triangleleft, glyph length=.95ex]},
bimr/.tip={Glyph[glyph math command=triangleright, glyph length=.95ex]},
}

\tikzset{
	tick/.style={postaction={
  	decorate,
    decoration={markings, mark=at position 0.5 with
    	{\draw[-] (0,.4ex) -- (0,-.4ex);}}}
  }
} 
\tikzset{
	slash/.style={postaction={
  	decorate,
    decoration={markings, mark=at position 0.5 with
    	{\draw[-] (.3ex,.3ex) -- (-.3ex,-.3ex);}}}
  }
} 

\newcommand{\upp}{\begin{tikzcd}[row sep=6pt]~\\~\ar[u, bend left=50pt, looseness=1.3, start anchor=east, end anchor=east]\end{tikzcd}}

\newcommand{\bito}[1][]{
	\begin{tikzcd}[ampersand replacement=\&, cramped]\ar[r, biml-bimr, "#1"]\&~\end{tikzcd}  
}
\newcommand{\bifrom}[1][]{
	\begin{tikzcd}[ampersand replacement=\&, cramped]\ar[r, bimr-biml, "{#1}"]\&~\end{tikzcd}  
}
\newcommand{\bifromlong}[2][]{
	\begin{tikzcd}[ampersand replacement=\&, column sep=#2, cramped]\ar[r, bimr-biml, "#1"]\&~\end{tikzcd}  
}

% Adjunctions
\newcommand{\adj}[5][30pt]{%[size] Cat L, Left, Right, Cat R.
\begin{tikzcd}[ampersand replacement=\&, column sep=#1]
  #2\ar[r, shift left=7pt, "#3"]
  \ar[r, phantom, "\scriptstyle\Rightarrow"]\&
  #5\ar[l, shift left=7pt, "#4"]
\end{tikzcd}
}

\newcommand{\adjr}[5][30pt]{%[size] Cat R, Right, Left, Cat L.
\begin{tikzcd}[ampersand replacement=\&, column sep=#1]
  #2\ar[r, shift left=7pt, "#3"]\&
  #5\ar[l, shift left=7pt, "#4"]
  \ar[l, phantom, "\scriptstyle\Leftarrow"]
\end{tikzcd}
}

\newcommand{\xtickar}[1]{\begin{tikzcd}[baseline=-0.5ex,cramped,sep=small,ampersand 
replacement=\&]{}\ar[r,tick, "{#1}"]\&{}\end{tikzcd}}
\newcommand{\xslashar}[1]{\begin{tikzcd}[baseline=-0.5ex,cramped,sep=small,ampersand 
replacement=\&]{}\ar[r,tick, "{#1}"]\&{}\end{tikzcd}}



  
  % amsthm %
\theoremstyle{definition}
\newtheorem{definitionx}{Definition}[chapter]
\newtheorem{examplex}[definitionx]{Example}
\newtheorem{remarkx}[definitionx]{Remark}
\newtheorem{notation}[definitionx]{Notation}


\theoremstyle{plain}

\newtheorem{theorem}[definitionx]{Theorem}
\newtheorem{proposition}[definitionx]{Proposition}
\newtheorem{corollary}[definitionx]{Corollary}
\newtheorem{lemma}[definitionx]{Lemma}
\newtheorem{warning}[definitionx]{Warning}
\newtheorem*{theorem*}{Theorem}
\newtheorem*{proposition*}{Proposition}
\newtheorem*{corollary*}{Corollary}
\newtheorem*{lemma*}{Lemma}
\newtheorem*{warning*}{Warning}
%\theoremstyle{definition}
%\newtheorem{definition}[theorem]{Definition}
%\newtheorem{construction}[theorem]{Construction}

\newenvironment{example}
  {\pushQED{\qed}\renewcommand{\qedsymbol}{$\lozenge$}\examplex}
  {\popQED\endexamplex}
  
 \newenvironment{remark}
  {\pushQED{\qed}\renewcommand{\qedsymbol}{$\lozenge$}\remarkx}
  {\popQED\endremarkx}
  
  \newenvironment{definition}
  {\pushQED{\qed}\renewcommand{\qedsymbol}{$\lozenge$}\definitionx}
  {\popQED\enddefinitionx} 

    
%-------- Single symbols --------%
	
\DeclareSymbolFont{stmry}{U}{stmry}{m}{n}
\DeclareMathSymbol\fatsemi\mathop{stmry}{"23}

\DeclareFontFamily{U}{mathx}{\hyphenchar\font45}
\DeclareFontShape{U}{mathx}{m}{n}{
      <5> <6> <7> <8> <9> <10>
      <10.95> <12> <14.4> <17.28> <20.74> <24.88>
      mathx10
      }{}
\DeclareSymbolFont{mathx}{U}{mathx}{m}{n}
\DeclareFontSubstitution{U}{mathx}{m}{n}
\DeclareMathAccent{\widecheck}{0}{mathx}{"71}


%-------- Renewed commands --------%

\renewcommand{\ss}{\subseteq}

%-------- Other Macros --------%


\DeclarePairedDelimiter{\present}{\langle}{\rangle}
\DeclarePairedDelimiter{\copair}{[}{]}
\DeclarePairedDelimiter{\floor}{\lfloor}{\rfloor}
\DeclarePairedDelimiter{\ceil}{\lceil}{\rceil}
\DeclarePairedDelimiter{\corners}{\ulcorner}{\urcorner}
\DeclarePairedDelimiter{\ihom}{[}{]}

\DeclareMathOperator{\Hom}{Hom}
\DeclareMathOperator{\Mor}{Mor}
\DeclareMathOperator{\dom}{dom}
\DeclareMathOperator{\cod}{cod}
\DeclareMathOperator{\idy}{idy}
\DeclareMathOperator{\comp}{com}
\DeclareMathOperator*{\colim}{colim}
\DeclareMathOperator{\im}{im}
\DeclareMathOperator{\ob}{Ob}
\DeclareMathOperator{\Tr}{Tr}
\DeclareMathOperator{\el}{El}




\newcommand{\const}[1]{\texttt{#1}}%a constant, or named element of a set
\newcommand{\Set}[1]{\mathsf{#1}}%a named set
\newcommand{\ord}[1]{\mathsf{#1}}%an ordinal
\newcommand{\cat}[1]{\mathcal{#1}}%a generic category
\newcommand{\Cat}[1]{\mathbf{#1}}%a named category
\newcommand{\fun}[1]{\mathrm{#1}}%a function
\newcommand{\Fun}[1]{\mathit{#1}}%a named functor




\newcommand{\id}{\mathrm{id}}
\newcommand{\then}{\mathbin{\fatsemi}}

\newcommand{\cocolon}{:\!}


\newcommand{\iso}{\cong}
\newcommand{\too}{\longrightarrow}
\newcommand{\tto}{\rightrightarrows}
\newcommand{\To}[2][]{\xrightarrow[#1]{#2}}
\renewcommand{\Mapsto}[1]{\xmapsto{#1}}
\newcommand{\Tto}[3][13pt]{\begin{tikzcd}[sep=#1, cramped, ampersand replacement=\&, text height=1ex, text depth=.3ex]\ar[r, shift left=2pt, "#2"]\ar[r, shift right=2pt, "#3"']\&{}\end{tikzcd}}
\newcommand{\Too}[1]{\xrightarrow{\;\;#1\;\;}}
\newcommand{\from}{\leftarrow}
\newcommand{\ffrom}{\leftleftarrows}
\newcommand{\From}[1]{\xleftarrow{#1}}
\newcommand{\Fromm}[1]{\xleftarrow{\;\;#1\;\;}}
\newcommand{\surj}{\twoheadrightarrow}
\newcommand{\inj}{\rightarrowtail}
\newcommand{\wavyto}{\rightsquigarrow}
\newcommand{\lollipop}{\multimap}
\newcommand{\imp}{\Rightarrow}
\renewcommand{\iff}{\Leftrightarrow}
\newcommand{\down}{\mathbin{\downarrow}}
\newcommand{\fromto}{\leftrightarrows}
\newcommand{\tickar}{\xtickar{}}
\newcommand{\slashar}{\xslashar{}}
\newcommand{\card}{\,^{\#}}


\newcommand{\inv}{^{-1}}
\newcommand{\op}{^\tn{op}}

\newcommand{\tn}[1]{\textnormal{#1}}
\newcommand{\ol}[1]{\overline{#1}}
\newcommand{\ul}[1]{\underline{#1}}
\newcommand{\wt}[1]{\widetilde{#1}}
\newcommand{\wh}[1]{\widehat{#1}}
\newcommand{\wc}[1]{\widecheck{#1}}
\newcommand{\ubar}[1]{\underaccent{\bar}{#1}}



\newcommand{\bb}{\mathbb{B}}
\newcommand{\cc}{\mathbb{C}}
\newcommand{\nn}{\mathbb{N}}
\newcommand{\pp}{\mathbb{P}}
\newcommand{\qq}{\mathbb{Q}}
\newcommand{\zz}{\mathbb{Z}}
\newcommand{\rr}{\mathbb{R}}


\newcommand{\finset}{\Cat{Fin}}
\newcommand{\smset}{\Cat{Set}}
\newcommand{\smcat}{\Cat{Cat}}
\newcommand{\catsharp}{\Cat{Cat}^{\sharp}}
\newcommand{\ppolyfun}{\mathbb{P}\Cat{olyFun}}
\newcommand{\ccatsharp}{\mathbb{C}\Cat{at}^{\sharp}}
\newcommand{\ccatsharpdisc}{\mathbb{C}\Cat{at}^{\sharp}_{\tn{disc}}}
\newcommand{\ccatsharplin}{\mathbb{C}\Cat{at}^{\sharp}_{\tn{lin}}}
\newcommand{\ccatsharpdisccon}{\mathbb{C}\Cat{at}^{\sharp}_{\tn{disc,con}}}
\newcommand{\sspan}{\mathbb{S}\Cat{pan}}
\newcommand{\en}{\Cat{End}}

\newcommand{\List}{\Fun{List}}
\newcommand{\set}{\tn{-}\Cat{Set}}




\newcommand{\yon}{\mathcal{y}}
\newcommand{\poly}{\Cat{Poly}}
\newcommand{\dir}{\Set{Dir}}
\newcommand{\rect}{\Set{Rect}}
\newcommand{\polycart}{\poly^{\Cat{Cart}}}
\newcommand{\hh}{\mathcal{h}}
\newcommand{\ppoly}{\mathbb{P}\Cat{oly}}
\newcommand{\0}{\textsf{0}}
\newcommand{\1}{\tn{\textsf{1}}}
\newcommand{\U}{\tn{\textsf{U}}}
\newcommand{\tri}{\mathbin{\triangleleft}}
\newcommand{\triright}{\mathbin{\triangleright}}
\newcommand{\tripow}[1]{^{\tri #1}}
\newcommand{\indep}{\Fun{Indep}}
\newcommand{\duoid}{\Fun{Duoid}}
\newcommand{\jump}{\pi}
\newcommand{\jumpmap}{\ol{\jump}}
\newcommand{\founds}{\Yleft}


% lenses
\newcommand{\biglens}[2]{
     \begin{bmatrix}{\vphantom{f_f^f}#2} \\ {\vphantom{f_f^f}#1} \end{bmatrix}
}
\newcommand{\littlelens}[2]{
     \begin{bsmallmatrix}{\vphantom{f}#2} \\ {\vphantom{f}#1} \end{bsmallmatrix}
}
\newcommand{\lens}[2]{
  \relax\if@display
     \biglens{#1}{#2}
  \else
     \littlelens{#1}{#2}
  \fi
}



\newcommand{\qand}{\quad\text{and}\quad}
\newcommand{\qqand}{\qquad\text{and}\qquad}


\newcommand{\coto}{\nrightarrow}
\newcommand{\cofun}{{\raisebox{2pt}{\resizebox{2.5pt}{2.5pt}{$\setminus$}}}}

\newcommand{\coalg}{\tn{-}\Cat{Coalg}}

\newcommand{\bic}[2]{{}_{#1}\Cat{Comod}_{#2}}

% ---- Changeable document parameters ---- %

\linespread{1.1}
\allowdisplaybreaks
\setsecnumdepth{section}
\settocdepth{section}
\setlength{\parindent}{15pt}
\setcounter{tocdepth}{1}



%--------------- Document ---------------%
\begin{document}

\title{Polynomial functors and entropy}

\author{David I. Spivak}

\date{\vspace{-.2in}}

\maketitle

\begin{abstract}
Past work shows that one can associate a notion of Shannon entropy to a Dirichlet polynomial $d\coloneqq a_nn^\yon+\cdots+a_22^\yon+a_11^\yon+a_00^\yon$, regarded as an empirical distribution. Indeed, entropy can be extracted from any $d\in\dir$ by a two-step process, where the first step is a rig homomorphism out of $\dir$, the \emph{set} of Dirichlet polynomials, with rig structure given by standard addition and multiplication. In this short note, we show that this rig homomorphism can be upgraded to a rig \emph{functor}, when we replace the set of Dirichlet polynomials by the category of ordinary (Cartesian) polynomials%
\footnote{Ren\`{e} Decartes at least invented the notation, e.g.\ $\yon^2+3\yon+2$, for polynomials.}

This time the process has three steps. The first step in the process is a rig functor $\polycart\to\poly$ sending a polynomial $p$ to $\dot{p}\yon$, where $\dot{p}$ is the derivative of $p$. The second is a rig functor $\poly\to\smset\times\smset\op$, sending a polynomial $q$ to the pair $(q(1),\Gamma(q))$, where $\Gamma(q)=\poly(q,\yon)$ can be interpreted as the global sections of $q$ viewed as a bundle, and $q(1)$ is its base. The last step, as for Dirichlet polynomials, is simply to extract the entropy as a real number from a pair of sets $(A,B)$; it is given by $\log A-\frac{\log B}{A}$. In summary, the entropy of $p$ is given by $\log \dot{p}(1)-\frac{\log\Gamma(\dot{p}\yon)}{\dot{p}(1)}$.
\end{abstract}

In \cite{spivak2021dirichlet} the authors---myself and Tim Hosgood---present a process for extracting the Shannon entropy from an empirical distribution, e.g.\ the following: 
\begin{equation}\label{eqn.empirical}
  \begin{tikzpicture}[scale=0.5]
    \node at (-2,2.5) {draws};%$\dot{p}(1)\cong8\cong$};
    \begin{scope}
      \draw[rounded corners] (0,0) rectangle ++(6,5);
      \foreach \y in {1,2,3,4}
        \draw[thick,black,fill=gray] (1,\y) circle (2mm);
      \foreach \x in {2,3,4,5}
        \draw[thick,black,fill=gray] (\x,1) circle (2mm);
    \end{scope}
    \draw[thick,-Latex] (3,-0.5) to (3,-1.5);
     \node at (3.75,-1) {$\pi$};
   \node at (-2,-3) {outcomes};%$p(1)\cong5\cong$};
    \begin{scope}[shift={(0,-4)}]
      \draw[rounded corners] (0,0) rectangle ++(6,2);
      \foreach \x in {1,2,3,4,5}
        \draw[thick,black,fill=white] (\x,1) circle (2mm);
    \end{scope}
  \end{tikzpicture}
\end{equation}
The picture in \eqref{eqn.empirical} represents a set of five (5) possible \emph{outcomes} and eight (8) empirical \emph{draws}: four of the draws were from the first outcome, and 
the rest were evenly distributed among the remaining four outcomes. This picture corresponds to a Dirichlet polynomial $d\coloneqq 1\cdot 4^\yon+4\cdot 1^\yon$ in the sense that $d(1)\cong 8$ and $d(0)\cong 4$. Moreover, $d$ can be regarded as a functor $\smset\op\to\smset$ and the function $\pi$ is $d$ applied to the unique function $0\to 1$. We showed that the entropy of any finite empirical distribution%
\footnote{In the case of \eqref{eqn.empirical}, one can calculate the entropy as $H(p)=H(\{\frac{1}{2},\frac{1}{8},\frac{1}{8},\frac{1}{8},\frac{1}{8})=2$
}
can be extracted from a certain rig homomorphism $\dir\to\rect$, where $\dir$ is the rig (in $\smset$) of Dirichlet polynomials under $+$ and $\times$, and $\rect$ is a rig that was constructed (ad hoc) for this purpose.%
\footnote{For example, addition in $\rect$ is given by sums and weighted geometric means:
\[(A_1,W_1)+(A_2,W_2)=(A_1+A_2,(W_1^{A_1}W_2^{A_2})^{\frac{1}{A_1+A_2}}).\]
Luckily, we will not need to look at that formula again.}

There is a duality \cite[Section 4]{spivak2020dirichlet} between Dirichlet polynomials and ordinary (Cartesian) polynomials, under which $1\cdot 4^\yon+4\cdot 1^\yon$ is sent to $1\cdot\yon^4+4\cdot\yon^1$, which we write simply as $p\coloneqq\yon^4+4\yon$. For the above bundle, we find $5\cong p(1)$ and $8\cong\dot{p}(1)$, where $\dot{p}=4\yon^3+4$ is the derivative of $p$ with respect to $\yon$. But now the map $\pi$ is inherent in $p$ and cannot simply be extracted as it was above for Dirichlet polynomials. Moreover, the map $p\mapsto\dot{p}$ is \emph{not} functorial in $p$, nor is there generally any comparison map between $p$ and $\dot{p}$. Thus while the story Hosgood and I told about Shannon entropy for Dirichlet polynomials was also technically viable for Cartesian polynomials, we decided that the story was cleaner and seemed to have better formal properties in the Dirichlet setting.

However, there is a fix that we were not aware of, and this fix makes the Cartesian story much more attractive than the Dirichlet story. It is more attractive because now the \emph{function} $\dir\to\rect$ is replaced by a \emph{functor} $\polycart\to\smset\times\smset\op$. Both the former and the latter are rig maps, i.e.\ they both preserve a notion of addition and multiplication. Certainly the category $\smset\times\smset\op$ is not ad hoc, though one needs to define a symmetric monoidal product on it
\[
	(A_1,B_1)\otimes(A_2,B_2)\coloneqq\left(A_1A_2\,,\,B_1^{A_2}B_2^{A_1}\right),
\]
previous uses of which I am not aware. This monoidal product $\otimes$ distributes over the coproduct, making $\smset\times\smset\op$ a distributive monoidal category, and in particular a \emph{rig category}.

There are two pieces left to describe: the functor $\polycart\to\smset\times\smset\op$ and the function $\ob(\smset\times\smset\op)\To{h}\rr$ that extracts the entropy. The latter is the following, somewhat strange-looking formula:
\[
h(A,B)\coloneqq\log A - \frac{\log B}{A}.
\]
The point is that this is merely a final extraction phase, whereas our interest is in the rig functor $\polycart\to\smset\times\smset\op$. It can be factored into two parts:
\[
\polycart\To{p\mapsto\dot{p}\yon}\poly\To{q\mapsto\big(q(1),\Gamma(q)\big)}\smset\times\smset\op
\]
We will explain these two functors in the main section, in particular the fact that each preserves the rig structure; for now we merely note that $\Gamma(q)\coloneqq\poly(q,\yon)$.

For now, we conclude our summary by simply giving the full composite: the function that takes a polynomial $p$ and returns its entropy is given by
\[
  H(p)\coloneqq\log \dot{p}(1)-\frac{\log\Gamma(\dot{p}\yon)}{\dot{p}(1)}
\]
For example, consider the polynomial $p=\yon^4+4\yon$, depicted in \eqref{eqn.empirical}. Then we calculate
\[
\dot{p}\yon=4\yon^4+4\yon
,\quad
\dot{p}(1)=8
,\quad
\Gamma(\dot{p}\yon)=4^4*1^4=2^8
,\qand 
H(p)=\log 8-\frac{\log 2^8}{8}=2
\]

The paper is quite short, and the introduction was written so that people familiar with $\poly$ may not need to read further. For less familiar readers, we include two additional sections: a section giving the relevant background on $\poly$ and $\polycart$, and a section proving the main results.

\chapter{Background}

%---- Section ----%
\section{Basics}


The main purpose of this section is to fix notation and provide a brief overview of polynomial functors in one variable. More extensive background material can be found in \cite{spivak2022poly} and \cite{kock2012polynomial}. 

\begin{definition}[Polynomial functor]\label{def.poly}
Given a set $S$, we denote the corresponding representable functor by
\[\yon^S\coloneqq\smset(S,-)\colon\smset\to\smset,\]
e.g. $\yon^S(X)\coloneqq X^S$. In particular $\yon=\yon^1$ is the identity and $\yon^0=1$ is constant singleton.

A \emph{polynomial functor} is a functor $p\colon\smset\to\smset$ that is isomorphic to a sum of representables, i.e.\ for which there exists a set $T$, a set $p[t]\in\smset$ for each $t\in T$, and an isomorphism
\[
p\cong\sum_{t\in T}\yon^{p[t]}.
\]
We call $T$ the set of \emph{$p$-types}, and for each type $t\in T$ we call $p[t]$ the set of \emph{$p$-terms of type $t$}.%

A \emph{morphism} $\varphi\colon p\to p'$ of polynomial functors is simply a natural transformation between them. It is called \emph{cartesian} if for every map of sets $f\colon S\to S'$, the naturality square
\[
\begin{tikzcd}
  p(S)\ar[r, "p(f)"]\ar[d, "\varphi(S)"']&p(S)\ar[d, "\varphi(S')"]\\
  p'(S')\ar[r, "p'(f)"']&p'(S')\ar[ul, phantom, very near end, "\lrcorner"]
\end{tikzcd}
\]
is a pullback of sets. We denote the category of polynomial functors by $\poly$ and the wide subcategory of polynomials and cartesian maps by $\polycart$.\end{definition}

For any polynomial $p=\sum_{t\in T}\yon^{p[t]}$, we have a canonical isomorphism $p(1)\cong T$; hence from now on we will denote $p$ by
\begin{equation}\label{eqn.poly_notation}
p=\sum_{I\in p(1)}\yon^{p[I]}
\end{equation}
so that the $p$-types are written with upper-case letters, e.g. $I\in p(1)$, and its terms are written with corresponding lower-case letters, e.g. $i\in p[I]$.

\begin{remark}\label{rem.positions_and_directions}
Using the Yoneda lemma, we can understand a morphism $p\to q$ in $\poly$ to consist of two parts $(\varphi_1,\varphi^\sharp)$ as follows:
\begin{equation}\label{eqn.mapsharp}
  \varphi_1\colon p(1)\to q(1)
  \qqand
  \varphi^\sharp_I\colon q[J]\to p[I],
\end{equation}
where $J\coloneqq\varphi_1(I)$. That is, $\varphi_1$ is a function from $p$-types to $q$-types, and $\varphi^\sharp_i$ is a function on terms that \emph{depends on a choice of position $I\in p(1)$}. We refer to $\varphi_1$ as the \emph{on-types function} and to $\varphi^\sharp$ as the \emph{backwards on-terms} function.

One can check that a map $\varphi\colon p\to q$ is cartesian iff the backwards-on-terms function $\varphi^\sharp_I$ is a bijection $p[I]\cong q[\varphi_1I]$ for each type $I\in p(1)$.
\end{remark}

\begin{example}[Positions and global sections, $\Gamma$]\label{ex.pos_glob}
For any polynomial $p$, we will be particularly interested in two sorts of maps: $\yon\to p$ and $p\to\yon$. The former is easy: a map $\yon\to p$ is given on types by choosing a single type $I\in p(1)$ to be the image of the unique type $!\in\yon(1)$ and backward on terms there is a unique choice of function $p[I]\to 1=\yon[!]$. Thus we have $p(1)\cong\poly(\yon,p)$.

More interesting are the maps $\gamma\colon p\to\yon$. This time $\gamma$ is trivial on types: each type $I\in p(1)$ is sent to the unique type $!\in\yon(1)$. However on terms, we need a map $\varphi^\sharp_I\colon 1\to p[I]$ for each $I$, meaning a choice of term $i\in I$ for each $I$. In other words, writing $\Gamma(p)\coloneqq\poly(p,\yon)$, we have
\begin{equation}\label{eqn.global}
\Gamma(p)\cong\prod_{I\in p(1)}p[I].
\end{equation}
We refer to $\Gamma(p)$ as the set of \emph{global sections} of $p$, as is justified by the bundle terminology the next section.

Note that $-(1)\colon\poly\to\smset$ and $\Gamma\colon\poly\to\smset\op$ are functorial, as they are represented and corepresented by $\yon\in\poly$. We will be very interested in the functor
\begin{equation}\label{eqn.fundamental}
F\colon\poly\to\smset\times\smset\op
\end{equation}
given by $F(p)\coloneqq (p(1),\Gamma(p))$. In fact, $F$ is a left adjoint, but we do not need that for this paper.
\end{example}

%\begin{definition}[Polynomial functor]\label{def.poly}
%Given a set $S$, we denote the corresponding representable functor by
%\[\yon^S\coloneqq\smset(S,-)\colon\smset\to\smset,\]
%e.g. $\yon^S(X)\coloneqq X^S$. In particular $\yon=\yon^1$ is the identity and $\yon^0=1$ is constant singleton.
%
%A \emph{polynomial functor} is a functor $p\colon\smset\to\smset$ that is isomorphic to a sum of representables, i.e.\ for which there exists a set $I$, a set $p[i]\in\smset$ for each $i\in I$, and an isomorphism
%\[
%p\cong\sum_{i\in I}\yon^{p[i]}.
%\]
%We call $I$ the set of \emph{$p$-positions}, and for each type $i\in I$ we call $p[i]$ the set of \emph{$p$-directions at $i$}.%
%
%A \emph{morphism} $\varphi\colon p\to p'$ of polynomial functors is simply a natural transformation between them. A morphism is called \emph{cartesian} if for every map of sets $f\colon S\to S'$, the naturality square
%\[
%\begin{tikzcd}
%  p(S)\ar[r, "p(f)"]\ar[d, "\varphi(S)"']&p(S)\ar[d, "\varphi(S')"]\\
%  p'(S')\ar[r, "p'(f)"']&p'(S')\ar[ul, phantom, very near end, "\lrcorner"]
%\end{tikzcd}
%\]
%is a pullback of sets. We denote the category of polynomial functors by $\poly$ and the wide subcategory of polynomials and cartesian maps by $\polycart$.
%\end{definition}
%
%For any polynomial $p=\sum_{i\in I}\yon^{p[i]}$, we have a canonical isomorphism $p(1)\cong I$; hence from now on we will denote $p$ by
%\begin{equation}\label{eqn.poly_notation}
%p=\sum_{i\in p(1)}\yon^{p[i]}.
%\end{equation}
%
%\begin{remark}\label{rem.positions_and_directions}
%Using the Yoneda lemma, we can understand a morphism $p\to q$ in $\poly$ to consist of two parts $(\varphi_1,\varphi^\sharp)$ as follows:
%\begin{equation}\label{eqn.mapsharp}
%  \varphi_1\colon p(1)\to q(1)
%  \qqand
%  \varphi^\sharp_i\colon q[j]\to p[i],
%\end{equation}
%where $j\coloneqq\varphi_1(i)$. That is, $\varphi_1$ is a function from $p$-positions to $q$-positions, and $\varphi^\sharp_i$ is a function on directions that \emph{depends on a choice of position $I\in p(1)$}. We refer to $\varphi_1$ as the \emph{on-positions function} and to $\varphi^\sharp$ as the \emph{backward on-directions} function.
%
%One can check that a map $\varphi\colon p\to q$ is cartesian iff the backward-on-directions function $\varphi^\sharp_i$ is a bijection $p[I]\cong q[\varphi_1I]$ for each type $I\in p(1)$.
%\end{remark}

\section{Derivatives and bundles}

We can understand polynomial functors in terms of bundles, using the derivative. For any polynomial $p$, its derivative $\dot{p}$ is defined as follows:
\[
\dot{p}\coloneqq\sum_{I\in p(1)}\sum_{i\in p[I]}\yon^{p[I]-\{i\}}
\]
where $p[I]-\{i\}$ denotes the set-difference. Note that $\dot{p}(1)\cong\sum_{I\in p(1)}p[I]$ is the set of all $p$-terms, and it comes with a map $\dot{p}(1)\to p(1)$ to the set of $p$-types. This map of sets---which we call a bundle---is often understood to be the polynomial itself. A map of polynomials $\varphi\colon p\to q$ can be written in terms of these bundles:
\[
\begin{tikzcd}
	\dot{p}(1)\ar[d]&p(1)\times_{q(1)}\dot{q}(1)\ar[l, "\varphi^\sharp"']\ar[r]\ar[d]&\dot{q}(1)\ar[d]\\
	p(1)\ar[r, equal]&p(1)\ar[r, "\varphi_1"']&q(1)\ar[ul, phantom, very near end, "\lrcorner"]
\end{tikzcd}
\]
Again, $\varphi$ is cartesian if $\varphi^\sharp$ is a bijection.

\begin{remark}
Note that $\dot{p}$ is not functorial in $p\in\poly$. Indeed, there exists a unique map $\yon\to\yon^2$, but there does not exist a map $1\to 2\yon$.

There is also no reasonable comparison map between $\dot{p}$ and $p$ in general. It simply cannot go $\dot{p}\to p$ since there is no map $1\to\yon$, and if we take it to go $p\to\dot{p}$ then which map $\yon^2\to 2\yon$ should we use? Instead we use the following.
\end{remark}

\begin{proposition}
The assignment $E(p)\coloneqq\dot{p}\yon$ is a functor $E\colon\polycart\to\polycart$, and it is equipped with a map $E(p)\to p$, natural in $p\in\polycart$.
\end{proposition}
\begin{proof}
We can think of $E(p)\coloneqq\dot{p}\yon$ as follows:
\begin{equation}\label{eqn.derivy}
\dot{p}\yon\cong\sum_{I\in p(1)}\sum_{i\in p[I]}\yon^{p[I]}
\end{equation}
Given a cartesian map $\varphi\colon p\to q$, the bijection $\varphi^\sharp\colon q[\varphi_1(I)]\cong p[I]$ lets us define a map $\dot{p}\yon\to\dot{q}\yon$ in an obvious way. The required map $\dot{p}\yon\to p$ is cartesian and is given on types by $(I,i)\mapsto I$; it is clearly natural in $p\in\polycart$.
\end{proof}

\section{Rig monoidal structure}

The category $\poly$ has coproducts $p+q$ and products $p\times q$ given by usual polynomial arithmetic. We will be more interested in the former:%
\footnote{
The only reason we introduce $\times$ for $\poly$ is that we will be writing $\dot{p}\yon$, which is shorthand for $\dot{p}\times\yon$, throughout the paper. We will not see other products of polynomials, though products of sets will come up quite often.
}
coproducts constitute a symmetric monoidal product with unit $0$. A type in $p+q$ is a type in $p$ or disjointly a type in $q$, and a term of that type is as specified in $p$ or $q$, respectively.

We will also be interested in another monoidal product called \emph{Dirichlet product} and denoted $-\otimes-$; the types and terms of $p\otimes q$ are given by the following formula:
\begin{equation}\label{eqn.dir_formula}
  \left(\sum_{I\in p(1)}\yon^{p[I]}\right)\otimes
  \left(\sum_{J\in q(1)}\yon^{q[J]}\right)\coloneqq
  \sum_{(I,J)\in p(1)\times q(1)}\yon^{p[I]\times q[J]}.
\end{equation}
This gives a symmetric monoidal structure $(\poly,\yon,\otimes)$. A type in $p\otimes q$ is just a pair of types $(I,J)\in p(1)\times q(1)$ and a term of it is just a pair of terms $(i,j)\in p[I]\times q[J]$.

In terms of bundles, $p+q$ and $p\otimes q$ are respectively given by
\[
\begin{tikzcd}
	\dot{p}(1)+\dot{q}(1)\ar[d]&\dot{p}(1)\times\dot{q}(1)\ar[d]\\
	p(1)+q(1)&p(1)\times q(1)
\end{tikzcd}
\]
i.e. $\dot{(p+q)}(1)\cong\dot{p}(1)+\dot{q}(1)$ and $\dot{(p\otimes q)}(1)\cong\dot{p}(1)\times\dot{q}(1)$.

The $\otimes$-structure distributes over the $+$ structure:
\[
p\otimes (q_1+q_2)\cong (p\otimes q_1)+(p\otimes q_2),
\]
thus making $(\poly,0,+,\yon,\otimes)$ a distributive monoidal category, and in particular a rig monoidal category.

\begin{remark}
Some readers may be interested in the Leibniz rule, that $\dot{(p\times q)}\cong\dot{p}\times q+p\times\dot{q}$, where $\times$ is the categorical product in $\poly$. This holds, but we will not need it in this paper.
\end{remark}

\chapter{Main results}

We divide this section into two pieces. \cref{sec.CT} is the category theory, providing what seems to be a novel symmetric monoidal structure on $\smset\times\smset\op$ and showing that both $p\mapsto\dot{p}\yon$ and $q\mapsto(q(1),\Gamma(q))$ are rig functors. At the end of this section, we will have a rig functor $\polycart\to\smset\times\smset\op$.

\cref{sec.entropy} is the finishing step, providing a function $\smset\times\smset\op\to\rr$ and showing that when it is combined with the above, the map $H\colon\ob(\polycart)\to\rr$ sends a polynomial $p$ the Shannon entropy of the empirical distribution defined by $p$.

\section{Category theory}\label{sec.CT}

Below we will often denote products merely by juxtaposition, both for polynomials $pq\coloneqq p\times q$ and for sets $AB\coloneqq A\times B$.

\begin{proposition}
The functor $\dot{p}\yon$ is a rig functor $\polycart\to\polycart$. 
\end{proposition}
\begin{proof}
Clearly $\dot{0}=0$ and $\dot{(p+q)}\cong\dot{p}+\dot{q}$, and by multiplying both sides by $\yon$ we see that $E$ preserves the coproduct structure. There is an isomorphism $\dot{\yon}\yon\cong\yon$, and for any $p,q\in\polycart$ there is also an isomorphism $\dot{(p\otimes q)}\yon\cong(\dot{p}\yon)\otimes(\dot{q}\yon)$, as follows from \eqref{eqn.derivy} and \eqref{eqn.dir_formula}; thus $p\mapsto\dot{p}\yon$ preserves the $\otimes$-structure. All of these isomorphisms are natural in $p,q\in\polycart$, completing the proof.
\end{proof}

The following corollary is straightforward, since $\polycart$ inherits $+$ and $\otimes$ from the forgetful functor $\polycart\to\poly$.

\begin{corollary}\label{cor.E}
The functor $E(p)\coloneqq\dot{p}\yon$ is a rig functor $E\colon\polycart\to\poly$. 
\end{corollary}

\begin{example}\label{ex.global_E}
For any polynomial $p$, we have
\[
\Gamma(\dot{p}\yon)\cong\prod_{I\in p(1)}p[I]^{p[I]}.
\]
This formula---which follows directly from \cref{eqn.global,eqn.derivy}---will be relevant when connecting the category theory to the entropy later on. 
\end{example}

\begin{proposition}
The category $\smset\times\smset\op$ has a distributive monoidal structure:
\begin{align*}
  (A_1,B_1)+(A_2,B_2)&\coloneqq(A_1+A_2,B_1B_2)\\
  (A_1,B_1)\otimes(A_2,B_2)&\coloneqq(A_1A_2,B_1^{A_2}B_2^{A_1})
\end{align*}
The units are $(0,1)$ and $(1,1)$ respectively.
\end{proposition}
\begin{proof}
Coproducts in $\smset\op$ are products in $\smset$, justifying the first line; these clearly form a symmetric monoidal structure. For the $\otimes$-monoidal structure, note that the formula is symmetric and that $(1,1)\otimes(A_2,B_2)\cong(A_2,B_2)$ as desired. Associativity is justified as follows:
\begin{align*}
  (A_1,B_1)\otimes((A_2,B_2)\otimes(A_3,B_3))&\cong
  %(A_1(A_2A_3),B_1^{A_2A_3}(B_2^{A_3}B_3^{A_2})^{A_1})\\\cong
	(A_1A_2A_3,B_1^{A_2A_3}B_2^{A_1A_3}B_3^{A_1A_2})\\&\cong
	((A_1,B_1)\otimes(A_2,B_2))\otimes(A_3,B_3).
\end{align*}
There is an absorption map $(0,1)\otimes (A,B)\cong(0,B)\to(0,1)$, and the distributivity of $\otimes$ over $+$ is justified as follows:
\begin{align*}
  (A,B)\otimes((A_1,B_1)+(A_2,B_2))&\cong 
  (A(A_1+A_2),B^{A_1+A_2}(B_1B_2)^A)\\&\cong
  (AA_1+AA_2,B^{A_1}B^{A_2}B_1^AB_2^A)\\&\cong
  ((A,B)\otimes(A_1,B_1))+((A,B)\otimes(A_2,B_2)).
\end{align*}
We leave the remaining details to the interested reader.
\end{proof}

\begin{proposition}\label{prop.F}
The functor $F\colon\poly\to\smset\times\smset\op$ from \eqref{eqn.fundamental} is a rig functor.
\end{proposition}
\begin{proof}
Recall that $F(p)\coloneqq(p(1),\Gamma(p))$. Clearly $0(1)=0$ and $(p+q)(1)\cong p(1)+q(1)$. Also $\Gamma(0)=1$ and $\Gamma(p+q)\cong\Gamma(p)\times\Gamma(q)$; hence $F$ preserves coproducts. Moreover, we have $\yon(1)=1$ and $(p\otimes q)(1)\cong p(1)\times q(1)$ and $\Gamma(\yon)= 1$, so to show that $F$ preserves the $\otimes$ structure, it remains only to show 
\[
  \Gamma(p\otimes q)\cong \Gamma(p)^{q(1)}\times\Gamma(q)^{p(1)}
\]
This is justified as follows:
\begin{align*}
	\Gamma(p\otimes q)&\cong
	\prod_{(I,J)\in p(1)\times q(1)}p[I]q[J]\\&\cong
	\left(\prod_{(I,J)\in p(1)\times q(1)}p[I]\right)\times
		\left(\prod_{(I,J)\in p(1)\times q(1)}q[J]\right)\\&\cong
	\prod_{J\in q(1)}\prod_{I\in p(1)}p[I]\times\prod_{I\in p(1)}\prod_{J\in q(1)}q[J]\\&\cong
	\Gamma(p)^{q(1)}\times\Gamma(q)^{p(1)}	
\end{align*}
\end{proof}

Noting that $(\dot{p}\yon)(1)\cong\dot{p}(1)$, we can summarize the section above. Namely, the functors $E\colon\polycart\to\poly$ and $F\colon\poly\to\smset\to\smset\op$ from \cref{cor.E,prop.F} compose to form a rig functor $\hh\coloneqq F\circ E$ given by
\begin{equation}\label{eqn.entropy_data}
\begin{aligned}
	\polycart&\To{\mathcal{h}}\smset\times\smset\op\\
	p&\mapsto (\dot{p}(1),\Gamma(\dot{p}\yon)).
\end{aligned}
\end{equation}
We refer to $\hh(p)$ as \emph{the entropy data} of the polynomial $p$.

\section{Shannon entropy}\label{sec.entropy}

We define a partial function $L\colon\ob(\smset\times\smset\op)\to\rr$ by
\begin{equation}\label{eqn.L}
(A,B)\mapsto \log A-\frac{\log B}{A}
\end{equation}
When $A=0$ and $B=1$, we define this function to be $L(0,1)=0$; for all other cases where $A=0$, $B=0$, or either $A$ or $B$ is infinite we leave $L(A,B)$ undefined. We will be only interested in this map when it is is composed with the entropy data $\mathcal{h}$ from \eqref{eqn.entropy_data}.

\begin{lemma}
Let $p\in\polycart$ with $(A,B)\coloneqq\hh(p)$. Then $B\neq 0$ and if $A=0$ then $B=1$.
\end{lemma}
\begin{proof}
We have $A\coloneqq\dot{p}(1)$ and $B\coloneqq\Gamma(\dot{p}\yon)$. One easily checks using \cref{ex.pos_glob} that for any $q\in\poly$, the set $\Gamma(q\yon)\neq0$ is nonempty since every $(q\yon)$-type has at least one term. If $\dot{p}(1)=0$ then $p=P\in\smset$ is constant, so $\dot{p}\yon=0$ as well, and $\Gamma(0)=1$.
\end{proof}

\begin{definition}[Empirical distribution]
Let $p\neq 0$ be a nonzero polynomial and suppose that the cardinality of $\dot{p}(1)\in\smset$ is finite, $\card\dot{p}(1)<\infty$. We define the \emph{empirical distribution defined by $p$} to be the following function $P\colon p(1)\to[0,1]$:
\[
 P(I)\coloneqq\frac{\card p[I]}{\card\dot{p}(1)}
\]
Note that $1=\sum_{I\in p(1)}P(I)$, so this is a probability distribution.
\end{definition}

Recall that the Shannon entropy $H(P)$ of a probability distribution $P\colon X\to[0,1]$ is given by
\[
H(P)\coloneqq-\sum_{x\in X}P(x)\log P(x)
\]

\begin{theorem}
Let $p\neq0$ be a nonzero polynomial with $\card\dot{p}(1)<\infty$, and let $P$ be the empirical distribution defined by $p$. Then we have
\[
H(P)=L\hh(p)
\]
where $H$ is the Shannon entropy and $L,\hh$ are as defined in \cref{eqn.entropy_data,eqn.L}.
\end{theorem}
\begin{proof}
We need to show that the following holds:
\[
H(P)=\log\dot{p}(1)-\frac{\log\Gamma(\dot{p}\yon)}{\dot{p}(1)}.
\]
With the fact $\Gamma(\dot{p}\yon)\cong\prod_{I\in p(1)}p[I]^{p[I]}$ from \cref{ex.global_E} in hand, this is a routine calculation:
\begin{align*}
	H(P)&\coloneqq
	-\sum_{I\in p(1)}\frac{\card p[I]}{\card\dot{p}(1)}\log \frac{\card p[I]}{\card\dot{p}(1)}\\&=
	\frac{1}{\card\dot{p}(1)}\sum_{I\in p(1)}\card p[I]\big(\log\card\dot{p}(1)-\log\card p[I]\big)\\&=
	\frac{1}{\card\dot{p}(1)}\left(\card\dot{p}(1)\log\card\dot{p}(1)-\log\prod_{I\in p(1)}\card p[I]^{\card p[I]}\right)\\&=
	\log\card\dot{p}(1)-\frac{\log\Gamma(\dot{p}\yon)}{\dot{p}(1)}
\end{align*}
\end{proof}

\printbibliography 
\end{document}
